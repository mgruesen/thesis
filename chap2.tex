\chapter{RELATED WORK}
\label{related}
% The related works section is comprised of the more current projects being
% developed in the problem domain addressed by the work in this thesis. 
Many
approaches to solving the numerous problems SDN brings about exist, yet there
is a lack of consensus on what is the ``correct'' way. This field of research
continues to grow, and overlap with other research interests is becoming more
common. The major sections in this chapter discuss the current SDN solutions,
heterogeneous computing, and event processing frameworks.

\section{Software Defined Networking}
\label{related:sdn}
SDN provides a networking device model where the control and data planes are
decoupled from one another; a single control plane can be responsible for
distributing application execution across a set of network switches viewed as
a unified data plane instance. Applications generally operate at the control
plane level and rely on distributed platforms to push the logic down to the
hardware level. Allowing users to define their own control and data plane
logic enables the creation and adoption of custom networking protocols. The
amount of control gained with respect to the shape and behavior of the network
being managed creates more intelligent networks.

% Work this in somewhere
% Not all networking applications operate in the same conceptual space, that is,
% they generally act as a part of a larger system. Applications can operate on a
% particular traffic flow, a node within a network, or across the entire network.

\subsection{OpenFlow}
\label{related:of}
The OpenFlow \cite{openflow} SDN model defines a switch that separates and
distributes the control and data planes. Centralized controllers aggregate
data plane switches and communicate over a dedicated channel utilizing the
OpenFlow protocol. The protocol defines messages for packet I/O, data plane
resource modifications, as well as state and configuration queries.

An OpenFlow switch's data plane is can be viewed as a flow table that defines
forwarding behavior for packets that match a particular set of fields within
a packet header. Packets that do not match entries in a flow table are sent
to the controller for further inspection. The controller is able to update flow
tables by installing new forwarding rules for packets that match ones sent
from the data plane.

The one downside to the OpenFlow model is the bottleneck created when utilizing
a messaging channel between two different physical devices. Freeflow focuses
on the implementation of a virtual switch where the control and data plane
logic are contained within the same device.

\subsection{DPDK}
\label{related:dpdk}
Intel's Data Plane Development Kit (DPDK) \cite{dpdk} provides a framework that
allows programmers to create highly optimized data plane applications. This
platform utilizes custom drivers that provide raw access to networking devices
and computational resources. The system is tuned to natively support Intel
brand hardware and allows users to create C applications that execute in
a runtime environment. 

Freeflow's port abstraction resides at a slightly higher level than DPDK to support more networking interfaces, and as such DPDK ports have been integrated into the FFVM.

\subsection{Netmap}
\label{related:netmap}
The netmap project \cite{netmap} is comprised of three functional components
contained in a single kernel module. With the netmap kernel module, user space
applications have access to network interface cards, the host operating system
networking stack, as well as VALE virtual ports and netmap ``pipes''. 

The broad class of network interfaces that netmap supports is desireable in the FFVM implementation. To compensate for the lack of control provided in the abstraction, netmap ports have been implemented in the Freeflow port abstraction.

\subsection{OpenDataPlane}
\label{related:odp}
The OpenDataPlane (ODP) project \cite{odp} establishes an open source API for
defining networking data plane applications. ODP provides application
portability over numerous networking architectures. Use of hardware
accelerators can be achieved without burdening the user with a required
knowledge of the capabilities and features present in an execution environment. 

ODP and the FFVM lie at roughly the same level with respect to high level language support and low level hardware abstractions. FFVM focuses more on flexibility and programmability with respect to data plane processing, whereas ODP focuses on exposing APIs to interface with other SDN execution environments.

\subsection{Open Virtual Switch}
\label{related:ovs}
Open Virtual Switch (OVS) \cite{ovs} enables the distribution of data
plane applications over an aggregation of switching devices. OVS replaces
the bridge between hyper-visors running on top of a distributed machine that
is typically provided by OS kernels. The OVS controller is capable of managing
pure software and/or hardware switches, and supports the usage of specialized
hardware accelerators. 

Freeflow applications can take advantage of the distribution of data plane switch logic by way of the FFVM. Integration with the OVS distributed architecture would mitigate northboud and southbound communication from the hardware to the application, and viceversa.

\section{Languages}
\label{related:lang}
SDN programming languages exemplify the features and capabilities programmers
want to be supported in an SDN application execution environment. These give
indications of what a suitable execution environment should provide in order
to support a large number of higher level languages. The following two sections
describe a couple of the more mature SDN programming languages referenced.

\subsection{P4}
\label{related:p4}
P4 \cite{p4} provides a network programming language for protocol independent
packet parsers. These parsers are akin to the decoding stages executed in
Freeflow packet processing pipelines, but executed on dedicated hardware. The
forwarding model for P4 uses match+action tables (similar to flow tables) to
match on ``parsed representations'' (PR) generated by the parsers. A PR is the
data structure which holds the parsing state, which gives access to protocol
header fields. 

Given the popularity of the P4 programming language, support is being considered in future implementations of the FFVM. Packet parsing units can be application-specific integrated circuits (ASICs), that are highly optimized for packet decoding operations and are utilizing the P4 language could take advantage of these types of processors available to an FFVM execution environment.

\subsection{POF}
\label{related:pof}
Protocol Oblivious Forwarding (POF) \cite{pof} defines a small ISA that can
be executed on POF SDN switches. The POF model represents a packet processing
pipeline as a series of matching tables that perform packet header decoding
when invoked. The decoded header fields are stored in meta data as an offset
and length pair, referred to as ``search keys''. 

This storage strategy is
similar to the FFVM context binding environment. Referencing field values as pairs of header offsets and lengths provides more efficient memory usage in the environment.

\section{Heterogeneous Computing}
\label{related:hcp}
Much of the research in the domain of SDN in regards to runtime environments
encompass many of the problems found in heterogeneous computing platforms.
These issues tend to revolve around the utilization of optimized hardware
computational devices. This relates to the needs of networking programming
where a data plane is expected to be able to offload certain computation to
hardware components, such as encryption, checksums, and even packet header
processors. Below is a listing of the relevant projects in this domain that
were studied during the design and implementation of the Freeflow system.

\subsection{Heterogeneous System Architecture}
\label{related:hsa}
Or HSA \cite{hsa}, provides a specification for creating a heterogeneous
system architecture that supports the execution of applications across a
variety of computational resources. Interfacing with these different computing
devices requires knowledge of each devices memory model and execution
semantics, and in order to program against them the system must provide a
uniform view that each device conforms to. Many of the core components defined in the HSA specification were considered during the implementation of the FFVM. Unified views of memory and time as well as concurrent thread communication attempt to follow this model to orchestrate multiple compuational devices or computing cores.

\subsection{CUDA}
\label{related:cuda}
NVIDIA's Compute Unified Device Architecture (CUDA) \cite{cuda} for General
Purpose Graphics Computing Units (GPGPUs) allows programmers to utilize a
graphics card to execute applications on a SIMD device. Each graphics card
hosts a different set of features and capabilities, such as the number of cores
available, memory architecture, and instruction sets. The CUDA library
enables developers to write a single instance of a program and deploy
it on a wide selection of supported GPUs. This is achieved by utilizing the
NVIDIA CUDA Compiler (NVCC) which translates the higher level source code
to the device's native machine code by way of an intermediate representation
(IR). The NVIDIA Virtual Machine (NVVM) IR is based on the Low Level Virtual
Machine (LLVM) IR used by the Clang compiler. Once the code has been lowered
to the NVVM IR it can be translated to the parallel thread execution virtual
machine and ISA, PTX. These instructions are executed natively by any CUDA
enabled GPU device. 

The narrowed view of CUDA with respect to heterogeneous computing is similar to the FFVM design in that a specialized hardware components that posses different capabilities, features, and instruction sets need to be considered during compilation.

\subsection{Open Compute Language}
\label{related:ocl}
Khronos Group's Open Compute Language (OpenCL) \cite{opencl} offers an open,
cross-platform standard for programming parallel devices. Supported devices are
personal computers, servers, and systems-on-a-chip (SOCs). The Standard
Portable IR (SPIR), based on LLVM IR, was evolved into a similarly open and
cross-platform language called SPIR-V. This IR utilizes LLVM as the target
device, and allows for device specific code to be generated from a single
application source. 

OpenCL focuses on the more general problem discussed in
the previous topic, supporting a larger class of computational devices. FFVM is similiar to OpenCL in attempting to solve the general problem of managing and optimizing the usage of different computing resources on a single system.

\subsection{NetASM}
\label{related:netasm}
The customized networking assembly language, NetASM \cite{netasm}, defines an
instruction set architecture (ISA) that tuned directly towards networking
programming. Instructions for modifying packet headers and state as well
as table and specialized operations are natively supported. In addition to the
networking oriented instructions the set also contains basic logical,
arithmetic, and control-flow operations. Applications written in NetASM can
model the execution environment as either a traditional Turing complete
register machine or an extended abstract machine developed by the authors.

The NetASM ISA approaches the problem of processing instructions natively in an execution environment at the machine level. We had considered this approach, explained in Chapter \ref{insn}, but abandoned the cause as re-implementing basic arithmetic, logical, and control flow operations to be interpreted at run time would negatively impact performance. In essence, the solution implements the MIPS ISA with some additional specialized network programming instructions.

\subsection{RISC V}
\label{related:riscv}
RISC-V \cite{riscv} specifies an extensible ISA, which encompasses the standard
arithmetic, logical and control-flow operations with room for a number of user-
defined instructions. Support for variable length instructions, multiple
address spaces (32, 64, 128 bit), and modern parallelization architectures are
also provided. RISC-V instruction execution requires the usage of RISC-V CPUs,
currently provided through an ISA simulator (Spike), or a VM running a modified
Linux kernel.

An extensible ISA that provides basic ALU operations as a core instruction set that is added to is a more fitting choice for implementing a network programming ISA. However, lack of support for executing RISC-V instructions in commericial processors prevents the approach from being embraced. FFVM instructions need to be executed on physical hardware in a native machine code, which is not currently possible with this ISA.

\section{Event Processing Frameworks}
\label{related:event}
One of the goals in SDN is to make networks more reactive, more intelligent.
Event processing frameworks provide many of the mechanisms required to implement
typical networking application behavior. The following sections describe two
processing frameworks considered, but the project has not reached the state
of maturity where either can be integrated.

\subsection{Seastar}
\label{related:seastar}
The Seastar project \cite{seastar}, is an open source C++ framework that
optimizes applications for modern architectural features supported in many high
performance, highly parallel servers. These features include the use of different
networking stacks, modern language features for concurrent programming, and low
overhead lock-free message passing between computing cores. The combination of
these features lets the framework host powerful event driven applications.
Support for Seastar warrants high end hardware that can make full use of the
framework and was deemed too exotic to integrate at this time.

Seastar's high perfomance execution is enabled by the usage of many computational and memory resources, rarely found in commercial hardware. In its current state, the FFVM has not yet reached the state of maturity to utilize this exotic event processing framework. Enterprise servers possess and abundance of processors and memory, and would be required to host Seastar applications. During consideration, compilation of the Seastar framework was unsuccessful due to the lack of hardware resources available in the testing hardware.

\subsection{Libevent}
\label{related:libevent}
Libevent \cite{libevent} exposes an API for executing function callbacks
triggered by file descriptor, signal, and timeout events to simplify the
development of event driven network server applications. Supporting numerous
polling mechanisms on different platforms (Linux, Mac, Windows) allows
applications to be highly portable and efficient. The dependency on file
descriptors makes supporting port objects that are not socket based difficult
and inefficient.

Integrating the libevent API with networking interfaces 