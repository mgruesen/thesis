\chapter{INTRODUCTION}
\label{intro}

\section{Motivation}

Networking architecture has become the focus of many facets in computing
with the increased reliance on data. As more and more devices become connected
and users consume and generate data at an ever-increasing rate, the way we
accomodate these changes in networking infrastructure has prompted the need
for change. The conventional network switch is a fairly static, rigid device
that utilizes two primary components, a control plane and data plane. In a
network switch the control plane is responsible for the configuration and
management of the data plane. Whereas the data plane is responsible for routing
network traffic through a switch, also known as the forwarding path. These
components have long been tightly coupled, where the control plane configures 
the forwarding path at boot time and relies on configuration scripts and
settings to orchestrate the applications, the network data they operate on, 
and the physical resources available. This model has been sustainable, but is 
quickly becoming a major roadblock in the evolution of networking switches. 
For example, in a large data center, any downtime results in a loss of revenue 
and the goal is to minimize the amount of time spent rebooting and
reconfiguring switches as much as possible. The reason for this management
model is due to the fact that data planes themselves lack the ability to be
programmable, that is, they are not reactive. Forwarding decisions for packets
rely on matching tables that map network traffic flows to some desired 
behavior. These match tables are defined and managed by the control plane, but
used by the data plane. When a table modification is needed, the control and
data planes must be torn down and rebuilt to reflect the changes made. In
order to overcome this hurdle control and data planes need to be programmable.

Software Defined Networking (SDN) provides a framework to address the
problems with conventional networking switches by introducing abstractions
to help describe the major components of networking devices. The most prominent
standard in this new networking paradigm is the OpenFlow (OF) [ref] protocol,
which loosens the grip the control plane has over the forwarding plane by 
viewing the data plane as an abstract machine. The abstraction describes the 
forwarding path as a packet processing machine, capable of receiving,
processing, and forwarding packets that enter a networking device. The two 
components communicate through a messaging protocol to relay information. This
protocol serves as an application binary interface (ABI) for network switch 
applications by providing a north-bound interface to the control plane, and a 
south-bound interface to the data plane.

Even with the establishment of an abstract packet processing machine, there
is still exist numerous problems with respect to programmability. The abstract 
machine identifies the necessary components needed to procure the desired 
functionality, but gives no standard model to program against. Network switch
vendors do not have to conform to any ``standard'' when manufacturing these
devices, and as a result there is a vast amount of proprietary hardware that
varies greatly between them. These proprietary components include hardware 
accelerators for particular network processing functions, such as validation 
(checksums), security (encryption/decryption), and packet processing (header
decoders, matching tables, FPGAs). Though these capabilites are generally 
available in most devices, their implementation is unclear. Parts of this 
functionality is handled in software, whereas others are offloaded at runtime 
to more well suited hardware components. The documentation for programming 
against these devices is sparse at best, and only adds the difficulties in 
attempts to accurately model how a networking application can be targeted to 
native hardware.

From all of these problems emerges the need for network switch applications
that can configure the control plane and also utilize the full potential of 
the underlying hardware available to the data plane. Currently there exists
no standard language capable of producing applications that address both of
these problem sets, yet this functionality is becoming highly desireable.
Reliance on vendor applications to fulfill the needs of an increasing customer
base breeds a sort of stagnation, where users must wait until a manufacturer
provides the flexibility and functionality desired. Generic end-user
applications rarely satisfy \emph{all} of the needs of a particular consumer, 
and the ability to create customized applications that perform efficiently 
would allow for users to build something that works for \emph{them}. This 
concept is akin to the creation of user applications for general purpose 
computing devices, CPUs, which support a large set of higher level languages 
that can be lowered to a well-defined interface that can utilize the 
capabilities and features present in a general purpose machine. In order to
create said customized networking applications, networking devices need to be
more open and programmable.

Though there exist numerous general purpose high level programming languages
that provide constructs and abstractions in order to create applications for
general purpose machines, there exist very few that attempt to tackle the
problem of providing basic networking programming requirements. Network
programming languages build on a primitive instruction set architectures,
(ISAs), that provide arithmetic, logical, and control flow instructions.
However in order to fulfill the requirements of the network programming domain
they need to accomodate networking constructs such as protocol headers and 
match tables. 

Protocol headers vary greatly, and the language must provide the means to 
represent and extract the fields within these headers efficiently. The main 
issue with protocol header extraction is that many of these fields do not 
align to standard ISA word sizes [refs], that is, they tend to not be byte 
(8-bit) aligned. Fields can be contained in 2-bit (e.g. Ethernet Type) or even 
48-bit regions (e.g. Ethernet MAC addresses). The same issue arises in the case
of matching tables, where the key used to identify entries in the table 
generally has a size equal to size of fields in a header.

It is clear that traditional networking methods are nearing the end of their
sustainability, and that there is a need to evolve networking infrastructure
into something that is more well suited for the current and future needs of
consumers. Networking infrastructure needs to be more flexible, and allow 
network administrators to shape and mold their systems based on their
particular needs. In order to achieve this next step there is a need for a
high level programming language that supports networking constructs and
a native architecture that can implement the needs of the language. This
architecture must provide the physical and logical resources required by
applications through an ABI, allowing developers to program against a much
more concrete networking device.

\section{Current Solutions}

\subsection{Languages}
In the domain of SDN many other researchers identify the need for a high level
programming language capable of encapsulating networking programming 
functionality that produces custom networking applications for networking 
devices. These implementations vary in terms of approach, as there is currently
no standard networking device abstraction that can be easily targeted. Some 
focus primarily on packet header decoders, while others focus more on how match
tables are represented and utilized. Below is a listing and description of 
some of the more well established SDN programming languages currently being
developed.
\begin{itemize}
\item \emph{P4} - Programming Protocol-Independent Packet Processors.
\item \emph{POF} - Protocol Oblivious Forwarding.
\item \emph{Frenetic}
\end{itemize}

\subsubsection{P4}

\subsubsection{POF}

\subsubsection{Frenetic}


\subsection{Architectures}
Even with the establishment of a suitable high level programming langauge for
networking applications, these applications need some target framework to 
target. In order to provide some uniformity in terms of architecture, the 
common approach is to create a runtime environment, or virtual machine, to 
execute networking applications. 
\begin{itemize}
\item \emph{DPDK} - Data Plane Development Kit.
\item \emph{ODP} - Open Data Plane.
\item \emph{OVS} - Open Virtual Switch, or OpenVSwitch.
\item \emph{HSA} - Heterogeneous System Architecture.
\end{itemize}

\subsubsection{DPDK}

\subsubsection{ODP}

\subsubsection{OVS}

\subsubsection{HSA}

\section{Contributions}

\subsection{Steve}
\subsubsection{Steve Applications}
\subsubsection{Steve Compiler}
\subsubsection{LLVM}


\subsection{Freeflow}
\subsubsection{Virtual Machine}
\subsubsection{Runtime Support Library}


\subsection{What is novel?}

\subsection{What was done better?}
\begin{itemize}
\item language that guarantees safety
\item utilize an extensible IR (llvm), intrinsics
\item adaptable runtime environment, optimizations
\subsection{What was discovered?}

\section{Successes \& Failures}

\subsection{What didn't work?}
Lots.
\subsection{Why didn't it work?}
It was wrong.