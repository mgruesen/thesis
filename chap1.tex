\chapter{INTRODUCTION}
\label{intro}




%\section{Problem Statement}

Networking architecture has become the focus of many facets in computing with 
an increased reliance on network data. As more and more devices become 
connected and users produce and consume data at an ever-increasing rate, the 
way we accomodate these needs in networking infrastructure has prompted the 
need for change. In order to service the needs of users, network engineers 
require more functionality from their equipment than basic switching and 
routing. Their networks need to be more intelligent, and allow for more 
powerful user generated network applications. Unfortunately conventional 
network switching devices are fairly static and rigid, and do not provide the 
means to network engineers to create custom applications that suit their 
particular needs. A network switch is composed of two high level components, a 
control plane and a forwarding, or data, plane. The control plane manages the 
configuration and state of the device, as well as compiled application 
tenancy; whereas the data plane is responsible for executing forwarding 
behavior over network traffic flows in the system. These components are 
tightly coupled which hinders how a network application might be able to 
abstract their functionality. At the upper level, control plane management is, 
more or less, a fairly open entity. Users can configure their networks and 
applications running on them within the confines of the interface exposed by 
that particular switch vendor. To apply changes made in the control plane, 
which may alter the configuration of the data plane, usually the device must 
be rebooted. Most changes made to resources used by the data plane and 
applications can only occur when the system is starting up. This has much to 
do with the fact that the applications running on these devices are created by 
the vendors themselves, who understand the underlying architecture in the 
system and are able to push the application logic down into the hardware. The 
result is high performance networking applications, that utilize hardware 
accelerators and are tuned for that particular system. However this comes at a 
great cost, flexibility. Network engineers are at the mercy of networking 
device manufacturers when they want to mold their network to suit their needs. 
The data plane in a network switch remains a black box, with features and 
capabilities varying greatly between vendors. This lack of transparency makes 
it difficult to model an abstract machine that can be targeted by networking 
applications. 

Research in the domain of Software Defined Networking (SDN) investigates the decoupling of the two planes, giving users the ability to create networking applications that can respond to changes in network traffic flows in real time. The emphasis in this model is to allow a single control plane to manage numerous data planes in one or many switches as a unified entity. Most of the work in this field revolves primarily around virtual network switches, building on concepts provided by the OpenFlow specification \ref{of}. We chose to focus on the implementation of a single programmable network device, where the control and data plane exist within the same system. This allowed us to produce a specification for an abstract machine for networking applications. The machine defines memory and object models, program execution semantics, a set of required operations, access to guaranteed resources, and the types of objects that it operates on and their behaviors. It must also be able to support a variety of high level programming languages as well as platforms and network processor architectures. Optimization support for specialized instruction execution and offloading compuation to hardware accelerators and/or co-processors is also a key requirement to keep performance on par. Languages are able to interact with the virtual machine through an application binary interface (ABI), that defines symbols and rules that dictate how they can utilize the runtime system. This approach has produced an instance of an abstract machine that provides the necessary resources and capabilities required to support network application execution.



% The conventional network switch is a fairly static, rigid device
% that utilizes two primary components, a control plane and data plane. In a
% network switch the control plane is responsible for the configuration and
% management of the data plane. Whereas the data plane is responsible for 
% routing
% network traffic through a switch, also known as the forwarding path. These
% components have long been tightly coupled, where the control plane configures
% the forwarding path at boot time and relies on configuration scripts and
% settings to orchestrate the applications, the network data they operate on,
% and the physical resources available. This model has been sustainable, but is
% quickly becoming a major roadblock in the evolution of networking switches.
% For example, in a large data center, any downtime results in a loss of 
% revenue
% and the goal is to minimize the amount of time spent rebooting and
% reconfiguring switches as much as possible. The reason for this management
% model is due to the fact that data planes themselves lack the ability to be
% programmable, that is, they are not reactive. Forwarding decisions for 
% packets
% rely on matching tables that map network traffic flows to some desired
% behavior. These match tables are defined and managed by the control plane, 
% but
% used by the data plane. When a table modification is needed, the control and
% data planes must be torn down and rebuilt to reflect the changes made. In
% order to overcome this hurdle control and data planes need to be 
% programmable.

% Software Defined Networking (SDN) provides a framework to address the
% problems with conventional networking switches by introducing abstractions
% to help describe the major components of networking devices. The most 
% prominent
% standard in this new networking paradigm is the OpenFlow (OF) [ref] protocol,
% which loosens the grip the control plane has over the forwarding plane by
% viewing the data plane as an abstract machine. The abstraction describes the
% forwarding path as a packet processing machine, capable of receiving,
% processing, and forwarding packets that enter a networking device. The two
% components communicate through a messaging protocol to relay information. 
% This
% protocol serves as an application binary interface (ABI) for network switch
% applications by providing a north-bound interface to the control plane, and a
% south-bound interface to the data plane.

% Even with the establishment of an abstract packet processing machine, there
% is still exist numerous problems with respect to programmability. The 
% abstract
% machine identifies the necessary components needed to procure the desired
% functionality, but gives no standard model to program against. Network switch
% vendors do not have to conform to any ``standard'' when manufacturing these
% devices, and as a result there is a vast amount of proprietary hardware that
% varies greatly between them. These proprietary components include hardware
% accelerators for particular network processing functions, such as validation
% (checksums), security (encryption/decryption), and packet processing (header
% decoders, matching tables, FPGAs). Though these capabilites are generally
% available in most devices, their implementation is unclear. Parts of this
% functionality is handled in software, whereas others are offloaded at runtime
% to more well suited hardware components. The documentation for programming
% against these devices is sparse at best, and only adds the difficulties in
% attempts to accurately model how a networking application can be targeted to
% native hardware.

% From all of these problems emerges the need for network switch applications
% that can configure the control plane and also utilize the full potential of
% the underlying hardware available to the data plane. Currently there exists
% no standard language capable of producing applications that address both of
% these problem sets, yet this functionality is becoming highly desireable.
% Reliance on vendor applications to fulfill the needs of an increasing 
% customer
% base breeds a sort of stagnation, where users must wait until a manufacturer
% provides the flexibility and functionality desired. Generic end-user
% applications rarely satisfy \emph{all} of the needs of a particular consumer,
% and the ability to create customized applications that perform efficiently
% would allow for users to build something that works for \emph{them}. This
% concept is akin to the creation of user applications for general purpose
% computing devices, CPUs, which support a large set of higher level languages
% that can be lowered to a well-defined interface that can utilize the
% capabilities and features present in a general purpose machine. In order to
% create said customized networking applications, networking devices need to be
% more open and programmable.

% Though there exist numerous general purpose high level programming languages
% that provide constructs and abstractions in order to create applications for
% general purpose machines, there exist very few that attempt to tackle the
% problem of providing basic networking programming requirements. Network
% programming languages build on a primitive instruction set architectures,
% (ISAs), that provide arithmetic, logical, and control flow instructions.
% However in order to fulfill the requirements of the network programming 
% domain
% they need to accomodate networking constructs such as protocol headers and
% match tables.

% Protocol headers vary greatly, and the language must provide the means to
% represent and extract the fields within these headers efficiently. The main
% issue with protocol header extraction is that many of these fields do not
% align to standard ISA word sizes [refs], that is, they tend to not be byte
% (8-bit) aligned. Fields can be contained in 2-bit (e.g. Ethernet Type) or 
% even
% 48-bit regions (e.g. Ethernet MAC addresses). The same issue arises in the 
% case
% of matching tables, where the key used to identify entries in the table
% generally has a size equal to size of fields in a header.

% It is clear that traditional networking methods are nearing the end of their
% sustainability, and that there is a need to evolve networking infrastructure
% into something that is more well suited for the current and future needs of
% consumers. Networking infrastructure needs to be more flexible, and allow
% network administrators to shape and mold their systems based on their
% particular needs. In order to achieve this next step there is a need for a
% high level programming language that supports networking constructs and
% a native architecture that can implement the needs of the language. This
% architecture must provide the physical and logical resources required by
% applications through an ABI, allowing developers to program against a much
% more concrete networking device.

\section{Goals}
It becomes clear that there is need for runtime support for compiled 
networking applications. The virtual machine needed to provide functionality 
for both parts of a network switch, the control and data planes, and allow 
applications to have control and access to these resources at the user level. 
In addition to servicing the needs of applications, the runtime needed to be 
targetable on a variety of architectures and utilize the physical and logical 
resources available. 

% It becomes clear that there is a need for high level programming language
% that supported extensions into the domain of network programming. 
% The language
% needs to be flexible enough to accomodate common programming language features
% and networking domain specific concepts, but also provide safety guarantees to
% curtail undefined behavior, as much as possible, and produce well formed
% applications. Also the language needs to allow the programmer to define
% behavior across both components in a switch, the control and data planes. This
% also helps define the requirements of the target machine for these networking
% applications. The runtime needs to be fully programmable and execute these
% programs on native hardware. By utilizing a virtual machine as the execution
% environment, the runtime is able to be mapped to a variety of architectures
% and take advantage of any and all hardware optimizations present.

\section{Contributions}
Freeflow is our software implementation of an intelligent network switch. This
system provides a framework on which compiled network applications can be
loaded and executed. The system has two major components needed by
applications, a runtime environment to target which we call the Freeflow
Virtual Machine (FFVM), and a system interface which applications can be
programmed against.

\subsection{Virtual Machine}
FFVM is a process that provides the resources necessary for network
applications to execute. These resources are:

\begin{itemize}
\item \emph{Ports} - The source of I/O for applications.
\item \emph{Tables} - Matching data structures that define forwarding behavior.
\item \emph{Packet Context} - Contextual information about a packet.
\item \emph{Action Exection} - Native FFVM instructions to be executed by the
runtime.
\item \emph{Memory} - Packet Context buffers.
\item \emph{Threading} - Infrastructure for modeling applications in various
threading architectures.
\end{itemize}

\subsection{Runtime Support Library}
Freeflow provides a runtime support library that Steve applications can 
leverage to execute as efficiently as possible. The library houses a collection
of system calls, exposed as external C functions, to allow applications to call
into the system during execution.

\section{Background}
Throughout the development of this project there were a few implementation
methods that did not end up panning out. Each approach has their own pros
and cons, and are highlighted in the follow sections.

\subsection{DPDK}
Intel's DPDK provided users with a framework on which data plane applications
could be built utilizing highly optimized constructs and device drivers. The
examples laid out in the documentation is promising, and the desire to have
low level access to underlying hardware in a system is more than satisfied
in this impelementation. However, there were a few road blocks on the path to
embracing this powerful data plane runtime environment. First and foremost, the
device drivers provided were specifically written for Intel brand hardware. If
there is no genuine Intel network inferface cards (NICs), a virtual machine
would be necessary in order to emulate their functionality and be able to take
advantage of the low level functionality, such as direct access to raw
ethernet packet data. At the time we were investigating this approach, DPDK was
more in it's infancy (release 1.0, 1.1), and though there was a decent amount
of documentation, the usage of their system was not very intuitive. Lastly
there was the issue of language support, DPDK can only build C applications
using GCC as C++ support is only available with the usage of Intel's C++
compiler. The combination of all these issues led us to stray away from
targeting DPDK as our runtime environment.

\subsection{RISC V}
Our next approach focused more on the interaction between the language and
runtime components of our system, with the hopes of being able to execute
the language in a native ISA. We came across an extensible ISA, RISC V, that
provided not only basic arithmetic, logical, and control flow instructions but
also left room for a number of custom instructions. This seemed like a great
way to incorporate specialized network programming instructions into a native
format. Unfortionately, this would also require that we interpret and implement
all of the supported instructions in the set in an virtual machine. As a result
the execution took a serious hit in terms of performance at the cost of this
flexibility. This project requires a fine balance of flexibility and
performance, and the deviation between these properties was too great.

\section{Road Map}
This paper covers our initial work towards a fully programmable networking device. The topics covered are:
\begin{itemize}
\item Software Defined Networking - Chapter \ref{sdn}.
\item The Freeflow System Architecture - Chapter \ref{ff}.
\item The Freeflow Virutal Machine - Chapter \ref{vm}.
\item Runtime Support - Chapter \ref{rt}.
\item Experimentation - Chapter \ref{expr}.
\item Related Works - Chapter \ref{related}.
\end{itemize}