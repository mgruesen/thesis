\chapter{HARDWARE VIRTUALIZATION}
\label{hardware}
Initial research into the domain of SDN was based around the utilization
of frameworks providing low level access to networking resources. In order
to maintain high performance in user level application space, these devices
need to be virtualized. Network interfaces, both logical and physical, can 
expose a variety of capabilities and features that need to be encapsulated
to improve portability and optimal usage. The same is true for computational
devices, including CPUs, Network Processing Units (NPUs), and other various
hardware accelerators. 

The following sections detail the frameworks considered as well as the 
resulting implementation in the FFVM.

\section{DPDK}
\label{hardware:dpdk}
Intel's DPDK provides users with a framework on which data plane
applications can be built utilizing highly optimized constructs and
device drivers. The examples laid out in the documentation are
promising, and the desire to have low level access to underlying
hardware in a system is more than satisfied in this implementation.
However, there were a few road blocks on the path to embracing this
powerful data plane runtime environment. First and foremost, the
device drivers provided were specifically written for Intel brand
hardware. If there is no genuine Intel network interface cards
(NICs), a virtual machine would be necessary in order to emulate
their functionality and be able to take advantage of the low level
functionality, such as direct access to raw Ethernet packet data.
At the time we were investigating this approach, DPDK was more in
it's infancy (release 1.0, 1.1), and though there was a decent amount
of documentation, the usage of their system was not very intuitive.
Lastly there was the issue of language support, DPDK can only build
C applications using GCC as C++ support is only available with the
usage of Intel's C++ compiler. The combinations of these issues
contributed to a negative impact on portability, and would ultimately
lead us to investigate other approaches in the attempt to define
a suitable port abstraction.

\section{Netmap}
\label{hardware:netmap}

\section{Freeflow Ports}
\label{hardware:ffports}
