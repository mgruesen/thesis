\chapter{RELATED WORK}
\label{related}

There exist many approaches to dealing with the numerous problems present in
developing network applications. Below is a brief overview of the more mature implementation strategies currently being developed to solve these problems. In
particular we focus on the language and architecture components.

\section{Languages}
In the domain of SDN many other researchers identify the need for a high level
programming language capable of encapsulating networking programming 
functionality that produces custom networking applications for networking 
devices. These implementations vary in terms of approach, as there is currently
no standard networking device abstraction that can be easily targeted. Some 
focus primarily on packet header decoders, while others focus more on how match
tables are represented and utilized. Below is a listing and description of 
some of the more well established SDN programming languages currently being
developed. 
\begin{itemize}
\item \emph{P4} - Programming Protocol-Independent Packet Processors.
\item \emph{POF} - Protocol Oblivious Forwarding.
\item \emph{Frenetic}
\end{itemize}

\section{Architectures}
Even with the establishment of a suitable high level programming langauge for
networking applications, these applications need some target framework to 
target. In order to provide some uniformity in terms of architecture, the 
common approach is to create a runtime environment, or virtual machine, to 
execute networking applications. 
\begin{itemize}
\item \emph{DPDK} - Data Plane Development Kit.
\item \emph{ODP} - Open Data Plane.
\item \emph{OVS} - Open Virtual Switch, or OpenVSwitch.
\item \emph{HSA} - Heterogeneous System Architecture.
\end{itemize}